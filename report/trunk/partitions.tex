\section{Graph Partitioning for Distributed Memory}
\label{sec:partitions}

In this section, we will describe how the Hilbert curve is also
a convenient mechanism for paritioning locally connected graphs
that are embeddable in a low-dimensional space.  That is, it 
generates a partition with small edge cuts.  We will discuss how
the priority-dag scheduling approach enables us to decompose the 
problem into two phases.  The first phase is to extend \proc{Prism}
to support a reshuffling operator, given a priority value from each
vertex, which re-organizes the graph data structure in linear order 
according to the priority function.  The second phase is to partition
the $n$ vertices by merely assigning $n/p$-sized compact subintervals
of vertices to each of the $p$ multi-core nodes.  Finally, we 
will describe the software architecture that integrates MPI commands
communicating over edges spanning partitions with the priority-dag 
scheduled computations on each multi-core node.  We will test the
performance of our implementation by measuring strong-scaling
performance on a small set of test graphs.