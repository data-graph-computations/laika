\begin{abstract}
\label{abstract}
\emph{Data-graph computations} consist of a graph $G=(V,E)$, where
each vertex has data associated with it, and an update function 
which is applied to each vertex, taking as inputs the neighboring 
vertices.  \proc{Prism} is a framework for executing data-graph
computations in shared memory using a scheduling technique called 
\emph{chromatic scheduling}, where a coloring of the input graph
is used to parcel out batches of independent work, that is
sets of vertices with a common color, while preserving
determinism.  An alternative scheduling approach is \emph{priority-dag
scheduling} where a priority function $\rho$, mapping each vertex $v \in V$
to a real number, is used to orient the edges from low to high priority and
and thus generate a dag.  We propose to extend \proc{Prism} in two primary
ways.  First, we will extend it to use distributed memory to enable
problem sizes many orders of magnitude larger than the current 
implementation.  Second, we will replace the chromatic scheduler in
\proc{Prism} with a priority-dag scheduler and a priority function 
which generates a cache-efficient traversal of the vertices when the
input graph is locally connected and embeddable in a low-dimensional 
space.  This subset of graphs is important for the physical simulations
generated by the language Simit.  
\end{abstract}
